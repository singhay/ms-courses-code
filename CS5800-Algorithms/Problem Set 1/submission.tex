%%Do not change font size
\documentclass[12pt]{article}
\usepackage{amsthm,amssymb,fullpage}

\begin{document}

\begin{center}
\begin{tabular}{|c|}
\hline
Fall '15 CS 5800 Algorithms \hspace{5cm} Hackerrank Group: singhay\\
{\bfseries \large Problem Set 1}\\
Surekha Jadhwani, Akash Singh, and Ayush K. Singh\\
\hline
\end{tabular}
\end{center}

\section{Part 1}
\textbf{\large{Answers}}
\begin{enumerate}
\item
\textbf{Given:} A program P which computes the function $f(n)=1 for$ for all $n \in N$ and an arbitrary program Q.\\
\textbf{To Prove:} There exists another program R which determines whether P(n) = Q(n) for any input n.\\
<<<<<<< HEAD
\textbf{Proof:} By contradiction.\\
Assume that a Program R exists which could determine equivalence of two programs.\\
Now, consider a Program P which returns always 1 i.e. Program P is something like:\\
\texttt{
Program P\\
return 1;\\
end\\	}
Now consider a Program R which takes input program Q and returns whether or not the programs are equal.\\
\texttt{
	Program R(Program Q)\\
	return P == Q;\\
	end\\}
Say we pass program Q which works on an integer input into variable i and is as follows:\\
\texttt{
	Program Q(int i)\\
	if (i == 1)\\
	return 1;\\
	else\\
	loop forever;\\
	end\\}
Hence if a program Q in above format is passed, the program in many cases will never return an output. \\
In fact, any program which does not halt, if passed as an input, equivalence cannot be deduced. \\
For a program to be able to compute whether two programs are equal or not, the programs passed should always halt.\\
But, since we can never say whether a program will halt or not as per the proof of halting problem, we know that a program which can be used to check equivalence cannot exist.\\
\pagebreak
\item {\textbf{Modular Arithmetic}}
\begin{enumerate}
\item {In a modulo 13 system, what is the multiplicative inverse of 15?}\\
\textbf{Solution:} By Extended Euclidean Method\\
Let x be the multiplicative inverse of 15 in modulo 13 system.\\
=======
\textbf{Proof:} We know that P(n) computes f(n) = 1. \\
We can define program P as:\\
        \begin{quote}
        	define P(n):\\
		    \begin{quote}
		    return 1\\
		    \end{quote}
		\end{quote}    
Let the program Q(n) be defined as follows:\\
		\begin{quote}
		define Q(n):\\
			\begin{quote}
		    P(n)\\
		    return 1\\
		    \end{quote}
		\end{quote}    
		
Here, Program Q calls Program P(n) and returns 1 when P(n) halts for an input n. That is when Q(n) halts for an input otherwise it will run forever in infinite loop.\\
If P(n) = Q(n), then Q(n) returns 1 for all inputs. This implies that P(n) halts for any given input which solves the halting problem. \\
But, since, halting problem is undecidable, our assumption is contradicting.\\
Therefore, we cannot determine conclusively whether two programs are equal or not.  \\


\item {\textbf{Modular Arithmetic}}
\begin{enumerate}
\item {In a modulo 13 system, what is the multiplicative inverse of 15?}\\
\textbf{Solution:} Let \quad x be the multiplicative inverse of 15 in a modulo 13 system.\\
By Extended Euclidean Method\\
>>>>>>> origin/master
$ 13 = 15\cdot0 + 13\cdot1$\\
$ 15 = 13\cdot1 + 2\cdot1$\\
$ 13 = 2\cdot6 + 1$\\
$ 13 - 2\cdot6 = 1$\\
$ 13 - (15 - 13)\cdot6 = 13 + 13\cdot6 - 15 = 1$\\
$ 13\cdot\textbf{7} - 15 = 1$\\
$ \textbf{Answer:} \quad x = 7$

\item {To what number between 0 and 12 inclusive is the product $3\cdot5\cdot11\cdot17\cdot23\cdot29\cdot31\cdot47\cdot53$
congruent modulo 13? }\\
$\textbf{Solution:} x \equiv (3\cdot5\cdot11\cdot17\cdot23\cdot29\cdot31\cdot47\cdot53) mod 13$ \\
$ x mod 13 = (3\cdot5\cdot11\cdot17\cdot23\cdot29\cdot31\cdot47\cdot53) mod 13$\\
$ x mod 13 = ((3 mod 13)(5 mod 13)(17 mod 13)(11 mod 13)(23 mod 13)(29 mod 13)..\\
 \qquad ..(31 mod 13)(47 mod 13)(53 mod 13) mod 13)$\\
$ x mod 13 = (3\cdot5\cdot4\cdot11\cdot10\cdot3\cdot5\cdot8\cdot1)mod13$\\
$ x mod 13 = (15\cdot44\cdot30\cdot40)mod13$\\
$ x mod 13 = (2\cdot5\cdot4\cdot1)mod13$\\
$ x mod 13 = 40mod13$\\
$ \textbf{Answer:} \quad x = 1$

\item {Find the remainder when $19^{19}$ is divided by 11}\\
$\textbf{Precompute:} \quad 19^{1}mod11 = 8$ \hfill (Equation 1)\\
$19^{2}mod11 = (19^{1}mod11\cdot19^{1}mod11)mod11$\\
$ = (8\cdot8)mod11 = 9$ \hfill [(from Eq. 1)(Equation 2)]\\
$19^{4}mod11 = [19^{2}mod11\cdot19^{2}mod11]mod11 $\\
$ = (9\cdot9)mod11 = 4$ \hfill [(from Eq. 2)(Equation 3)]\\
$19^{8}mod11 = [19^{4}mod11\cdot19^{4}mod11]mod11 $\\
$ = (4\cdot4)mod11 = 5$ \hfill [(from Eq. 3)(Equation 4)]\\
$19^{16}mod11 = [19^{8}mod11\cdot19^{8}mod11]mod11 $\\
$ = (5\cdot5)mod11 = 3$ \hfill [(from Eq. 4)(Equation 5)]\\
\textbf{Solution:}We know that, $19^{19} = 19^{16}\cdot19^{2}\cdot19^{1}$
Therefore from Eq. 1, Eq. 2, and Eq. 3, We can write\\
$ 19^{19}mod11 = (19^{16}mod11\cdot19^{2}mod11\cdot19^{1}mod11)mod11$\\
$= (8\cdot9\cdot3)mod11 = (8\cdot(9\cdot3))mod11$\\
$= (8mod11\cdot27mod11)mod11$\\
$= (8\cdot5)mod11$\\
$= 40mod11$\\
$\textbf{Answer:} \quad 19^{19}mod11= 7$
\end{enumerate}
\pagebreak
\item { Solve the following using Mathematical Induction }
\begin{enumerate}
\item { For any integer $n \ge 0, 3|(2_{2n} - 1)$}\\
\textbf{To Prove:} $3|(2_{2n} - 1) i.e (2^{2n} - 1)/3 = k$, an integer OR $(2^{2n} - 1) = 3k$ \\
\textbf{Proof:} by Mathematical Induction \\
Let $P(n)$ be $(2^{2n} - 1) = 3k$ \\
\textbf{Basic Step:} for $n = 1$\\
$P(1): (2^{2*1} - 1) = 3$\\
\textbf{Induction Step:} Since $P(n)$ is true for $n = 1$, assuming its true for $P(n+1)$ \\
$P(n+1): (2^{2(n+1)} - 1) = (2^{2n}.2^{2} - 1)$\\
$P(n+1): [(3k + 1)2^{2} - 1] = 12k + 4 - 1 $\\
$P(n+1): 12k + 3 = 3(4k + 1)$
$P(n+1): = 3j$ \hfill [where j = 4k + 1]\\
Hence Proved.

\item {$\displaystyle{ 1 + \frac{1}{\sqrt{2}} + \frac{1}{\sqrt{3}} + \cdots +  \frac{1}{\sqrt{n}} \le 2\sqrt{n}}$, for all $n \ge 1$}\\
\textbf{To Prove:} {$\displaystyle{ 1 + \frac{1}{\sqrt{2}} + \frac{1}{\sqrt{3}} + \cdots +  \frac{1}{\sqrt{n}} \le 2\sqrt{n}}$, for all $n \ge 1$}\\
\textbf{Proof:} By Mathematical Induction \\
Let $P(n)$ be {$\displaystyle{1 + \frac{1}{\sqrt{2}} + \frac{1}{\sqrt{3}} + \cdots +  \frac{1}{\sqrt{n}} \le 2\sqrt{n}}$}\\
\textbf{Basic Step:} For $n = 1$\\
$P(1)$:  $\displaystyle{1 + \frac{1}{1}  \le 2\sqrt{1}}$\\
$P(1)$: $\displaystyle{1 \le 1}$\\
\textbf{Induction Step:} Since $P(n) $ is true for $ n = 1$, assuming its true for $P(n)$\\
Solving for $P(n+1)$\\
$P(n+1)$: {$\displaystyle{1 + \frac{1}{\sqrt{2}} + \frac{1}{\sqrt{3}} + \cdots +  \frac{1}{\sqrt{n}} + \frac{1}{\sqrt{n+1}} \le 2\sqrt{n+1}}$}\\
Replacing value for $P(n)$ in above expression\\
$P(n+1)$: {$\displaystyle{2\sqrt{n} + \frac{1}{\sqrt{n+1}} \le 2\sqrt{n+1}}$}\\
Multiplying both sides by $\displaystyle{\sqrt{n+1}}$\\
$P(n+1)$: {$\displaystyle{2\sqrt{n}\sqrt{n+1}+ 1 \le 2(n+1)}$}\\
$P(n+1)$: {$\displaystyle{2\sqrt{n^2 + n} + 1 \le 2n+2}$}\\
Subtracting 1 from both sides\\
$P(n+1)$: {$\displaystyle{2\sqrt{n^2 + n}  \le 2n+1}$}\\
Squaring both sides\\
$P(n+1)$: {$\displaystyle{4(n^2 +n) \le (2n+1)^2}$}\\
$P(n+1)$: {$\displaystyle{4n^2 +4n \le 4n^2+4n+1}$}\\
Canceling out common terms\\
$P(n+1)$: $\displaystyle{0 \le 1}$\\
Hence Proved.\\
\pagebreak
\item {Let n and k be non-negative integers with $n \ge k.$} \\
\textbf{To Prove:} $\displaystyle{\sum^{n}_{i=k} {i \choose k} = {n+1 \choose k+1}}$, for $n,k \ge 0$ and $n \ge k$\\
\textbf{Proof:} By Mathematical Induction\\
Let $P(n)$ be $\displaystyle{\sum^{n}_{i=k} {i \choose k} = {n+1 \choose k+1}}$\\
We know that $\displaystyle{{n \choose k} = {\frac{n!}{(n-k)!k!}}}$\\
\textbf{Basic Step:} For k=1,n=2\\
$P(1)$: $\displaystyle{\sum^{2}_{i=1} {i \choose 1} = {2+1 \choose 1+1}}$\\
$P(1)$: $\displaystyle{{1 \choose 1} + {2 \choose 1}  = {3 \choose 2}}$\\
$P(1)$: $\displaystyle{{\frac{1!}{0!1!}}+ {\frac{2!}{1!1!}} = {\frac{3!}{2!1!}}}$\\
$P(1)$: $\displaystyle{{\frac{1}{1}} + {\frac{2}{1}} = {\frac{6}{2}}}$\\
$P(1)$: $ 3 = 3$\\
\textbf{Inductive Step:} Since $P(n)$ is true for $n = 1$, assuming its true for $P(n)$\\
Solving for $P(n+1)$\\
$P(n+1)$: $\displaystyle{\sum^{n+1}_{i=k} {i \choose k} = {n+1+1 \choose k+1}}$\\
$P(n+1)$: $\displaystyle{\sum^{n}_{i=k} {i \choose k} + {n+1 \choose k} = {n+2 \choose k+1}}$\\
Replacing value for $P(n)$ in above expression\\
$P(n+1)$: $\displaystyle{{n+1 \choose k+1} + {n+1 \choose k} = {n+2 \choose k+1}}$\\
$P(n+1)$: $\displaystyle{{\frac{(n+1)!}{(n+1-k-1)!(k+1)!}} + {\frac{(n+1)!}{(n+1-k)!k!}} = {\frac{(n+2)!}{(n+2-k-1)!(k+1)!}}}$\\  
$P(n+1)$: $\displaystyle{{\frac{(n+1)!}{(n-k)!(k+1)!}} + {\frac{(n+1)!}{(n+1-k)!k!}} = {\frac{(n+2)!}{(n+1-k)!(k+1)!}}}$\\\\
Simplifying by expanding factorial and equalizing denominators, we get\\
$P(n+1)$: $\displaystyle{\frac{(n-k+1)(n+1)!}{(n-k+1)(n-k)!(k+1)k!} + \frac{(k+1)(n+1)!}{(n-k+1)(n-k)!(k+1)k!}}$\\ = $\displaystyle{\frac{(n+2)(n+1)!}{(n-k+1)(n-k)!(k+1)k!}}$\\ \\
Canceling out common terms\\
$P(n+1)$: $(n-k+1) + (k+1) = (n+2)$\\
$P(n+1)$: $n+2 = n+2 \Rightarrow 0 = 0$\\
Hence Proved.

\end{enumerate}
\pagebreak
\item Functions in ascending order of growth rate \\
\textbf{Solution:} Applying logarithms and simplifying given functions, we get:
\begin{center}
	\begin{tabular}{| l | c | r |}
		\hline
		No. & \textbf{g(x)} & \textbf{log(g(x)}) \\ \hline		
		1 & $n^{101/100}$ & 1.01$\cdot$log $n$ \\ \hline
		2 & $n\cdot2^{n+1}$ & log $n$ + ($n$+1)(log 2) \\ \hline
		3 & $n(log $n$)^{3}$ & log $n$ + 3 $\cdot$ log (log $n$) \\ \hline
		4 & $n$ log $n$ & log $n$ + log log $n$ \\ \hline
		5 & $n^{log log n}$ & log log $n \cdot $log $n$  \\ \hline
		6 & log($n^{2n}$) & log log $n^{2n}$ \\ \hline
		7 & $n^{log n}$ & log $n \cdot$ log $n$ \\ \hline
		8 & $2^{n}$ & $n \cdot$ log 2 \\ \hline
		9 & $n\cdot2^{n}$ & log $n + n \cdot$ log 2 \\ \hline
		10 & $2^{\sqrt{logn}}$ & log 2 $\cdot sqrt{log n}$ \\ \hline
		11 & $2^{2^{n + 1}}$ & $2^{n+1} \cdot$ log 2 \\ \hline
		12 & $e^{e^{n}}$ & $e^{n} \cdot $(log $e$) \\ \hline
		13 & log($n!$) & log $n$ + log log $n$ \\ \hline
		14 & $e^{log n}$ & log $n \cdot $log $e$ \\ \hline
		15 & $2^{log(\sqrt{n})}$ & $\frac{1}{2} \cdot$ log 2 $\cdot$ log $n$ \\ \hline
		16 & $\sqrt{2^{log n}}$ & $\frac{1}{2} \cdot$ log 2 $\cdot$ log $n$  \\ \hline
		17 & $2^{n^{2}}$ & $n^{2} \cdot $log 2 \\ \hline
		18 & $n!$ & $n$ log $n$ \\ \hline
		19 & (log $n$)! & log $n \cdot$ log log $n$ \\ \hline
		20 & log log $n$ & log log log $n$ \\ 								
		\hline
	\end{tabular}
\end{center}
By comparing log$(g(x))$ values for all given $g(x)$ from the above table, we get can write the functions in ascending order of complexity as follows:\\
$g_{20} \textless
g_{10} \textless
g_{15} =
g_{16} \textless
g_{14} \textless
g_{1} \textless
g_{19} \textless
g_{5} \le
g_{13} \textless
g_{4} \le
g_{6} \textless
g_{3} \textless
g_{7} \textless
g_{8} \textless
g_{9} \textless
g_{2} \textless
g_{18} \textless
g_{17} \textless
g_{11} \textless
g_{12}$

\item {For a given two functions f, g and h. Decide whether each of the following statements
are correct and give a proof for each part}
\indent 
\begin{enumerate}
\item 
{If $f(n) = \Omega(g(n))$ and $g(n) = \Omega(f(n))$, then $f(n) = \Theta(g(n))$.} \\
\textbf{Solution:} The above statement is correct and the proof is as follows:\\            
\textbf{Given:}  $f(n) = \Omega(g(n))$, $g(n) = \Omega(f(n))$     \\       
\textbf{To Prove:} $f(n) = \Theta(g(n))$            \\
\textbf{Proof:} We have $f(n) = \Omega(g(n))$ \\
By definition of Big-Omega notation, it implies\\
$\exists$ c$_{1}$$\textgreater$0, n$_{0}$ $\ge$ 0: $\forall$ n$\textgreater$n$_{0}$:\\  
f(n) $\ge$ c$_{1}$g(n) 
\hfill(Equation 1)\\

Also, $g(n) = \Omega(f(n))$ \\
i.e., $\exists$ c$_{2}\textgreater$0, n$_{0}$ $\ge$0: $\forall$ n$\textgreater$n$_{0}$: $g(n) \ge$ c2$f(n)$  \\            
i.e., 1/c$_{2}$ * $(g(n)) \ge f(n)$ 
\hfill [As c$_{2}$ $\textgreater$ 0]\\                                          
Let 1/c$_{2}$ = c (some constant)\\
Therefore, c$g(n) \ge f(n)$ \\
\hfill (Equation 2)\\              
From Equations (1) and (2), we get \\
$\exists$ c$_{1}$$\textgreater$0, n$_{0}$ $\ge$0: $\forall$ n$\textgreater$n$_{0}$: \\
$f(n) \ge c_{1}g(n)$
\hfill (Equation 1)\\                         

Also, $g(n) = \Omega(f(n))$ \\
i.e., $\exists$ c$_{2}$$\textgreater$0, n$_{0}$ $\ge$0: $\forall$ n$\textgreater$n$_{0}$: $g(n) \ge$ c$2f(n)$              
i.e., 1/c$_{2}$ * $(g(n)) \ge f(n)$
\hfill [As c$_{2}$ $\textgreater$ 0]\\                                         
Let 1/c$_{2}$ = c (some constant) \\
Therefore, c$g(n) \ge f(n)$
\hfill (Equation 2)\\

From Equations (1) and (2), we get \\
c$g(n) \ge f(n) \ge c_{1}g(n)$ where c, c$_{1}\textgreater$0            
\hfill (Equation 3)\\        
We know that, $f(n) = \Theta(g(n)) $\\
iff  $\exists$ c$_{1}$,c$_{2} \textgreater$0, n$_{0}$ $\ge$0: $\forall$ n$\textgreater$n$_{0}$: c$_{1}$$g(n) \ge f(n)$ $\ge$ c$_{2}g(n)$
\hfill (Equation 4)\\

From equations (3) and (4), we conclude that given f(n) = $\Theta(g(n))$ if $f(n) = \Omega(g(n))$ and $g(n) = \Omega(f(n))$.
Hence, proved.		

\item 
{If $f(n) = (g(n))$, then $g(n) \not \in O(f(n))$}\\
\textbf{Solution:} The above statement is correct and the proof is as follows:\\            
\textbf{Given:}  $f(n) = o(g(n))$ \\       
\textbf{To Prove:} $g(n) \in O(f(n))$\\
\textbf{Proof:} We have $f(n) = o(g(n))$ \\ 
By definition of Little-o notation, it implies\\
$\forall$ c$_{1}$ $\textgreater$ 0: $\exists$n$_{0}\ge$0: $\forall$ n$\textgreater$n$_{0} f(n) \textless$ c$_{1}$$g(n)$   \hfill (Equation 1) \\                        
Let’s assume, $g(n) \in O(f(n))$ \\
i.e., $\exists$c$_{2}\textgreater$0, n$_{0}\ge$0: $\forall$ n$\textgreater$n$_{0}$:  
$g(n) \le$ c$_{2}f(n)$ \hfill (Equation 2)\\
We can see that Equation (2) contradicts with Equation (1).\\
Hence, Proving by contradiction $g(n) \not \in O(f(n))$ if $f(n) = o(g(n))$

\item { If f(n) = O(h(n)) and g(n) = O(h(n)), then f(n) + g(n) = O(h(n))}\\
\textbf{Solution:} The above statement is correct and the proof is as follows:\\            
\textbf{Given:}  $f(n) = O(h(n))$ and $g(n) = O(h(n))$\\       
\textbf{To Prove:} $f(n) + g(n) = O(h(n))$\\
\textbf{Proof:} We have $f(n) = O(h(n))$ \\
By definition, $\exists$ c$_{1}$$\textgreater$0, n$_{0}\ge$0: $\forall$ n$\textgreater$n$_{0}$: $f(n) \le$ c$_{1}$$h(n)$ \hfill (Equation 1)\\  
Also, $g(n) = O(h(n))$ \\
By definition, $\exists$ c$_{2}$$\textgreater$0, n$_{0}\ge$0: $\forall$ n$\textgreater$n$_{0}$: $f(n) \le$ c$_{2}$$h(n)$ \hfill (Equation 2)\\ 
L.H.S. = $ f(n) + g(n) \le $c$_{1}h(n)$ + c$_{2}h(n) $\hfill [From 1 and 2]\\
L.H.S. = (c$_{1}$ + c$_{2}$)$h(n)$\\
L.H.S. = c$h(n)$ \hfill [where c = c$_{1}$ + c$_{2}$]\\
Therefore, $f(n) + g(n) \le $ c$h(n)$ \\
i.e. $f(n) + g(n) = O(h(n))$ Hence Proved.\\


\item {If $f(n) = O(h(n))$ and $g(n) = O(h(n))$, then $f(n) \cdot g(n) = O(h(n))$} \\
\textbf{Solution:} The above statement is incorrect and the proof is as follows:\\            
\textbf{Given:}  $f(n) = O(h(n))$ and $g(n) = O(h(n))$\\       
\textbf{To Prove:} $f(n)\cdot g(n) = O(h(n))$\\
\textbf{Proof:} We have $f(n) = O(h(n))$ \\
By definition, $\exists$ c$_{1}$$\textgreater$0, n$_{0}\ge$0: $\forall$ n$\textgreater$n$_{0}$: $f(n) \le$ c$_{1}$$h(n)$ \hfill (Equation 1)\\  
Also, $g(n) = O(h(n))$ \\
By definition, $\exists$ c$_{2}$$\textgreater$0, n$_{0}\ge$0: $\forall$ n$\textgreater$n$_{0}$: $f(n) \le$ c$_{2}$$h(n)$ \hfill (Equation 2)\\ 
L.H.S. = $ f(n)\cdot g(n) \le $c$_{1}h(n)$$\cdot$c$_{2}h(n) $\hfill [From 1 and 2]\\
L.H.S. = (c$_{1}$ $\cdot$ c$_{2}$)$h(n)^{2}$\\
L.H.S. = c$h(n)^{2}$ \hfill [where c = c$_{1}$$\cdot$c$_{2}$]\\
Therefore, $f(n) \cdot g(n) \le $ c$h(n)^{2} \hfill [h(n) \neq h(n)^{2}]$\\
but $f(n)\cdot g(n) \textless O(h(n)^{2})$ Hence Statement is False.\\
\end{enumerate}
\item 
{Justify and give the asymptotic upper and lower bounds for each of the following recurrences}
\begin{enumerate}
\item {$ T(n) = 3T(n/2) + n/logn $} \\
\textbf{Solution:}  By Akra-Bazzi Method\cite{akba}.
\begin{center}
$ T(x) = g(x) + \Sigma^{k}_{i-1}a_{i}T(b_{i}x + h_{i}(x))$ for $x \ge x_{0}$
\end{center}
The conditions for usage are:
\begin{itemize}
	\item {sufficient base cases are provided}
	\item {$a_{i}$ and $b_{i}$ are constants for all $i$}
	\item {$a_{i}$ \textgreater for all $i$}
	\item {$ 0 \textless b_{i} \textless 1 $ for all $i$}
	\item {$\left|g(x)\right| \in O(x^{c})$, where c is a constant and O notates Big $O$ notation}
	\item {$\left| h_{i}(x) \right| \in  \displaystyle{O\left(\frac{x}{(\log x)^{2}}\right)}$ for all $i$}
	\item {$x_{0}$ is a constant}
\end{itemize}
The asymptotic behavior of T(x) is found by determining the value of p for which $ \displaystyle{\sum_{i=1}^{k} a_{i} b_{i}^{p} = 1}$ and plugging that value into the equation.
\begin{center}
	$ \displaystyle{T(x) \in \Theta \left( x^p\left( 1+\int_1^x \frac{g(u)}{u^{p+1}}du \right)\right)}$
\end{center}
Here, $a = 3,$ $b = \displaystyle{\frac{1}{2}},$ $g(u) = \displaystyle{\frac{n}{log n}},$ $p = ??$\\
Applying that in our situation, $ \displaystyle{3\cdot\left(\frac{1}{2}\right)^{p} = 1} \Rightarrow p = 1.58496$\\
Substituting in the above mentioned Equation, \\
\begin{center}
	$ \displaystyle{T(x) \in \Theta \left( x^{1.58496}\left( 1+\int_1^x \frac{u/log u}{u^{1.58496+1}}du \right)\right)}$
\end{center}
On Solving and Integrating the later part of above Equation for n we get,
\begin{center}
	$ \displaystyle{T(n) \in \Theta \left(n(log log n)\right)}$
\end{center}
Now we actually can't use the master method to solve this recurrence relation\cite{rr} because of non polynomial difference between $f(n)$ and $n^{log_{2}3}$. However, we can still derive an upper bound by finding a similar recurrence that is larger than T(n), analyze the new recurrence using the master method, and use the result as an upper bound for T(n).
$ T(n) = 3T(n/2) + n/logn \le 3T(n/2) + n$, so if we call $S(n)$ the function such that $S(n)=3T(n/2)+ n$, we know that $S(n)\ge T(n)$. We can apply the master method to he function $S(n)$: $n$ is $\Theta(n)$, so $S(n)$ is $\Theta(n^{log_{2}3})$. But, as $T(n)\le S(n)$, we can conclude that $T(n)$ is $O(n^{log_{2}3})$. Here we can use only $O$ and not $\Theta$ as the function $S(n) \ge T(n)$; As we were only able apply the master method indirectly, we could show a tight bound if we use the result from Akra-Bazzi computed above and compare former [(3T(n/2)] and latter [n/logn] parts we get, \\
\textbf{Answer:} Upper bound: $n^{log_{2}3}$, Lower Bound: $\left(n(log log n)\right)$

\item {$ T(n) = \sqrt{n}T(\sqrt{n}) + n  $}\\
\textbf{Solution:}  By Rolling-Unrolling Method.\\
$ T(n) = n^{1/2}T(n^{1/2}) + cn $\\
$= \displaystyle{n^{1/2}\left[ n^{1/4}T\left(n^{1/2}\right) + cn^{1/2} \right] + cn}$\\
$= \displaystyle{n^{1/2 + 1/4}T\left(n^{1/2}\right) + cn^{1/2 + 1/2} + cn}$\\
$= \displaystyle{n^{1/2 + 1/4}T\left(n^{1/8}\right) + cn^{1/4} + 2cn}$\\
$= \displaystyle{n^{1/2 + 1/4 + 1/8}T\left(n^{1/8}\right) + 3cn}$\\
$...$\\
$= \displaystyle{n^(1-1/2^{k}) T(n^{1/2^{k}}) + kcn }$\\
$=...$ let $k = $log log $n$ so $2^{k}$ = log $n$ and $n^{2-k} = n^{1/logn} = 2 ...$\\
$= \displaystyle{n/2T(2) + cn }$loglog $n$ \hfill ($O(n/2) \textless O(n$loglog$n))$\\
$= \displaystyle{\Theta(nloglogn)}$\\
\textbf{Answer:} Upper and Lower Bound: $nloglogn$
\pagebreak
\item {$ T(n) = 3T(n - 1)  $}\\
\textbf{Solution:} By Rolling-Unrolling Method.\\
Assuming base case $T(0)$ = constant c.
$ T(n) = 3T(n - 1)  $ \hfill (Equation 1) \\
Substituting T(n) with T(n-1) in Equation 1\\
$ = 3[ 3T((n - 1) - 1)] = 3^{2}T(n-2) $ \\
$ = 3[3^{2}T((n-2) - 1)] = 3^{3}T(n-3) $ \\
$\cdot$ \\
$\cdot$ \\
$= 3^{k}T(n - k) $\\
Let $k = n \rightarrow 3^{n}T(0)$\\
Since $T(0)$ is a constant by our assumption, so the complexity of the above recurrence relation turns out to be $\Theta(3^{n}).$ \\
\textbf{Answer:} Upper Bound and Lower Bound: $3^{n}$

\end{enumerate}

\end{enumerate}

\section{Part 2}
\textbf{\large{Group Hackerrank Name: singhay }}
\begin{enumerate}
	\item Surekha Jadhwani : surekha@ccs.neu.edu / jadhwani.s@husky.neu.edu
	\item Akash Singh : singhaka@ccs.neu.edu / singh.aka@husky.neu.edu
	\item Ayush K. Singh : singhay@ccs.neu.edu / singh.ay@husky.neu.edu
\end{enumerate}

 \begin{thebibliography}{1}
 	
 	\bibitem{akba} Akra-Bazzi Method \texttt{https://en.wikipedia.org/wiki/Akra-Bazzi$\textunderscore$method}
 	
 	\bibitem{rr} Recurrence relations not solvable by the master method \texttt{http://www.cs.cornell.edu/courses/cs3110/2011sp/recitations/rec19.htm} 2011: CS3110 Recitations.
 	
 \end{thebibliography}
\end{document}
